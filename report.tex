\documentclass[12pt]{report}

% Packages (keep sorted)
\usepackage{
    hyperref, % For PDF bookmarks
}

% PDF bookmarks setup (keep sorted)
\hypersetup{
    colorlinks=true,
    linkcolor=blue,
}

\begin{document}

% Info section

% TODO(sanjay): come up with something more creative
\title{RTX Project Report}

\author{
    Craig, Shale\\
    20371384\\
    \texttt{sakcraig@uwaterloo.ca}
    \and
    Klen, Alex\\
    XXXXXXXX\\
    \texttt{XXXX@uwaterloo.ca}
    \and
    Menakuru, Sanjay\\
    20374915\\
    \texttt{smenakur@uwaterloo.ca}
    \and
    Wei, Jonathan\\
    XXXXXXXX\\
    \texttt{XXXX@uwaterloo.ca}
}

\maketitle

\begin{abstract}
    Here is our awesome abstract
\end{abstract}

\tableofcontents

% NOTE(sanjay): uncomment these if we add any figures or tables
% \listoffigures
% \listoftables


\part{Introduction}

\part{Kernel Implementation}
% This part is at max 30 pages.

\chapter{Scheduler}
% Make sure to include pseudocode and testing, if appropriate.

\section{Description}

\section{Running Time Analysis}

\chapter{Memory Allocator}
% Make sure to include pseudocode and testing, if appropriate.

\section{Description}

\section{Running Time Analysis}

\section{Measurements}

\chapter{Message Passing}
% Make sure to include pseudocode and testing, if appropriate.

\section{Description}

\section{Running Time Analysis}

\section{Measurements}

\chapter{I/O}
% Make sure to include pseudocode and testing, if appropriate.

\section{UART output}

\section{UART input}

\chapter{Misc}
% Make sure to include pseudocode and testing, if appropriate.

\section{Bridge Layer}

\part{User-level Processes}

\chapter{Proc 1}

\chapter{Proc 2}

\part{Lessons Learned}
% They call this the lessons learned summary.
% 1-2 pages
% what you did do well, both technically and organizationally, and what you
% would do differently if you were to do it again

\end{document}
