\documentclass[12pt]{report}

% Packages (keep sorted)
\usepackage{
    algorithm,      % For algorithm environment
    algpseudocode,  % For pseudocode
    hyperref,       % For PDF bookmarks
}

% PDF bookmarks setup (keep sorted)
\hypersetup{
    colorlinks=true,
    linkcolor=blue,
}

\begin{document}

% Info section

% TODO(sanjay): come up with something more creative
\title{RTX Project Report}

\author{
    Craig, Shale\\
    20371384\\
    \texttt{sakcraig@uwaterloo.ca}
    \and
    Klen, Alex\\
    20372654\\
    \texttt{ayklen@uwaterloo.ca}
    \and
    Menakuru, Sanjay\\
    20374915\\
    \texttt{smenakur@uwaterloo.ca}
    \and
    Wei, Jonathan\\
    XXXXXXXX\\
    \texttt{XXXX@uwaterloo.ca}
}

\maketitle

\begin{abstract}
    Here is our awesome abstract
\end{abstract}

\tableofcontents
\listofalgorithms

% NOTE(sanjay): uncomment these if we add any figures or tables
% \listoffigures
% \listoftables


\part{Introduction}

\part{Kernel Implementation}
% This part is at max 30 pages.

\chapter{Scheduler}
% Make sure to include pseudocode and testing, if appropriate.

\section{Description}
    The scheduler is responsible for dividing processor time amongst all of the
    system's processes. It handles context switching between processes, deciding
    which process to schedule next, and preempting or blocking processes when
    appropriate. Processes can be in one of the following states: new, ready,
    running, blocked on memory, blocked on message, or unused. `New' is a state
    that processes are given after being initialized so that the scheduler can
    properly context switch to them for the first time. The running state
    signifies that the process is the only currently running process in the
    system. Processes which are ready are placed in a priority queue, and the
    scheduler will select the ready process with the highest priority to run
    next. Processes which are blocked on memory are put into another priority
    queue called the blocked-on-memory queue, and when memory is freed, the
    blocked process with the highest priority is unblocked. Processes that are
    blocked on a message remain blocked until they are sent a message.

    A process can be preempted when it frees memory or sends a message. If the
    process that it unblocks has a higher priority, then the currently running
    process is preempted.

    The ready queues are priority queues implemented using heaps. The decision
    was made to use a priority queue for all ready/blocked processes instead of
    a queue for each priority level because it allowed the system to support an
    arbitrary number of priorities without having the release\_processor runtime
    depend on the number of priorities.

    Pseudocode for process initialization is listed in Algorithm
    \ref{code:proc_init}.

    \begin{algorithm}
        \caption{Process Initialization Pseudocode}
        \label{code:proc_init}
        \begin{algorithmic}[1]
        \Function{k\_initProcesses}{}
            \State Initialize process-ready queue
            \State Initialize blocked-on-memory queue

            \For{each process p, index i}
                \State p.pid $\gets$ i
                \State p.stack $\gets$ \Call{acquireMemoryBlock}{} three times + memory block size \Comment{Use three memory blocks for each process stack and place stack pointer at end of third one}
                \State push initial required values onto stack
                \State set p.startLocation to appropriate process function location
                \Comment{Reserve a debug envelope for each process to be able to display a debug message even if there is no free memory in the system}
                \State p.debugEnvelope $\gets$ acquireMemoryBlock()
                \State p.state $\gets$ new
                \State add p to the process-ready queue
            \EndFor \\
            \State Initialize CRT proc
        \EndFunction
        \end{algorithmic}
    \end{algorithm}

    See Algorithm \ref{code:release_processor} for a pseudocode implementation
    for the kernel-level function k\_releaseProcessor. This function is called
    any time the current process should be preempted and a new process should be
    scheduled. The function takes a parameter that specifies what the reason is
    for the context switch. The release reason can be one of the following: a
    voluntary yield, a process priority change, memory was freed, the system is
    out of memory, a message was sent, or a message was received. The scheduling
    behavior can be different for some of these reasons.

\begin{algorithm}
    \caption{Release Processor Pseudocode}
    \label{code:release_processor}
    \begin{algorithmic}[1]
    \Function{k\_releaseProcessor}{releaseReason}
        \Comment{First we process the changed states of a few interrupt-driven systems here to avoid ISRs modifying kernel state while they are interrupting kernel methods}
        \State \Call{k\_processUartOutput}{}
        \State \Call{k\_processUartInput}{}
        \State \Call{k\_processDelayedMessages}{}

        \Comment{These variables are set depending on the release reason}
        \State define targetState, \Comment{The state to give the currently running process}
            \State \indent srcQueue, \Comment{The queue to schedule the next process 
            from}
            \State \indent dstQueue \Comment{The queue to insert the current process into}

        \If{releaseReason = out of memory}
            \State srcQueue $\gets$ process-ready queue
            \State dstQueue $\gets$ blocked-on-memory queue
            \State targetState $\gets$ blocked on memory
        \ElsIf{releaseReason = message was received}
            \If{the current process's mail queue is empty}
                \State srcQueue $\gets$ process-ready queue
                \State dstQueue $\gets$ NULL \Comment{Don't place in any queue}
                \State targetState $\gets$ blocked on message
            \Else
                \State srcQueue $\gets$ process-ready queue
                \State dstQueue $\gets$ process-ready queue
                \State targetState $\gets$ ready
            \EndIf
        \Else
            \State srcQueue $\gets$ process-ready queue
            \State dstQueue $\gets$ process-ready queue
            \State targetState $\gets$ ready

            \If{currentProcess is the null process}
                \Comment{Keep the null process out of the ready queue - handle it separately}
                \State dstQueue $\gets$ NULL
            \EndIf
        \EndIf

        \If{srcQueue $\ne$ NULL and srcQueue.size $>$ 0}
            \State nextProc $\gets$ \Call{srcQueue.pop}{}
        \Else
            \State nextProc $\gets$ the null process
        \EndIf

    \algstore{releaseProcessor}
    \end{algorithmic}
\end{algorithm}

\begin{algorithm}
    \begin{algorithmic}[1]
    \algrestore{releaseProcessor}

        \If{currentProcess $\ne$ NULL}
            \Comment{Save old process info}
            \State currentProcess.stack $\gets$ \Call{getStackPointer}{}
            \State currentProcess.state $\gets$ targetState
            \If{dstQueue $\ne$ NULL}
                \State \Call{dstQueue.add}{currentProcess}
            \EndIf
        \EndIf

        \State oldState $\gets$ nextProc.state
        \State nextProc.state $\gets$ running
        \State currentProcess $\gets$ nextProc
        \State \Call{setStackPointer}{nextProc.stack}
        \If{oldState = new}
            \Call{\_\_rte}{} \Comment{If first time, pop the exception stack frame}
        \EndIf
    \EndFunction
    \end{algorithmic}
\end{algorithm}

\section{Running Time Analysis}

    The runtime for k\_initProcesses is $O(n\log n)$, where n is the number of
    processes in the system. This is because each processes is inserted into the
    process-ready queue, which is implemented as a heap and therefore taking
    $O(\log n)$ to insert an element. This function could be trivially modified
    to achieve a runtime of $O(n)$ time by putting all of the heap elements in
    an array first, and then heapifying it afterwards. Since there is a
    small, fixed number of processes in the system, this small optimization
    wasn't completed since this function only runs once at system startup.

    The runtime for k\_releaseProcessor, with the assumption that the UART and
    delayed message processing at the beginning will typically take $O(1)$ time,
    takes $O(\log n)$ time, where n is the number of processes in the system. It
    is bounded by removing and inserting from the source and destination
    priority queues (heaps). This provides good performance, and we can have an
    arbitrary number of priorities in the system because a priority queue was
    used instead of a queue per priority level.

\chapter{Memory Allocator}
% Make sure to include pseudocode and testing, if appropriate.

\section{Description}
    The memory allocator was designed with two orthogonal layers. The first is
    called the `block layer', and is responsible for allocating and deallocating
    blocks at a low level. The second is called the `metadata layer'; this layer
    builds upon the work done by the block layer, and adds the ability to store
    metadata about blocks. We will describe both of these layers independently.

\subsection{Block Layer}
    The block layer has a conceptually simple role. It is responsible for
    managing a pool of contiguous memory. Given a start memory address, an
    end memory address, and a block size, the block
    layer must provide two pieces of functionality. It must support allocating
    a block from the pool, and it must support freeing a block back into the
    pool. One design constraint is that it must prefer reusing a previously
    allocated block that has been freed, rather than handing out a block that
    has never been allocated before. The reason for this constraint will
    become clear when discussing the metadata layer.

    See Algorithm \ref{code:mem_block} for a pseudocode implementation of this
    layer. Note that this layer returns an error code when it runs out of
    memory. Further, this layer does some rudimentary sanity checking but
    doesn't handle some obvious error conditions. For example, this layer allows
    blocks to be inserted into the free list twice (assuming a trivial linked
    list implementation). This is not an error; it is the responsibility of
    the metadata layer to never allow this to happen. Finally, it is important
    to point out that this layer automatically zeroes memory. This is slightly
    detrimental to performance, but allows user processes to use binary
    comparisons between structs without caring about garbage in the struct's
    padding. We deemed this convenience worth the slight performance overhead.

    \begin{algorithm}
        \caption{Block layer pseudocode}
        \label{code:mem_block}
        \begin{algorithmic}[1]

            \State $blockSize \gets 2^7$
            \Comment{Blocksize in bytes (configurable)}

            \State $startAddr \gets 0$
            \Comment{Start address of pool (inclusive)}

            \State $endAddr \gets 2^{16}$
            \Comment{End address of pool (exclusive)}

            \State $nextAddr \gets startAddr$
            \Comment{Next available address in pool}

            \State $freeList \gets \{\}$
            \Comment{List of freed blocks}\\

            \Function{alloc\_block}{$ $}
                \If {$|freeList| > 0$} \Comment{Check free-list}
                    \State $blk \gets freeList[0]$
                    \State $freeList = freeList - \{blk\}$
                    \State \Call{zero}{$blk$}
                    \State \Return $blk$
                \EndIf\\

                \If {$nextAddr + blockSize >= endAddr$}
                    \Comment{Out-of-memory}
                    \State \Return $-1$
                \EndIf\\

                \State $blk \gets nextAddr$
                \State $nextAddr \gets nextAddr + blockSize$
                \State \Call{zero}{$blk$}
                \State \Return $blk$
            \EndFunction\\
            \Function{free\_block}{$blk$}
                \If {$blk < startAddr \vee nextAddr <= blk \vee endAddr <= blk$}
                    \State \Return \Comment{Outside valid range}
                \EndIf\\

                \If {$blk - startAddr \not\equiv 0 \bmod{blockSize}$}
                    \State \Return \Comment{Unaligned}
                \EndIf\\

                \State $freeList = freeList \cup \{blk\}$
            \EndFunction
        \end{algorithmic}
    \end{algorithm}


\subsection{Metadata Layer}
    The metadata layer builds atop the block layer, and adds the ability for the
    allocator to store metadata about blocks. Since it is built upon the block
    layer, it must fit its metadata storage in blocks. To avoid a costly setup
    phase, the metadata layer uses a lazy, arena-based storage scheme. The
    general version of the metadata layer supports an arbitrary amount of
    metadata per block ranging from some non-zero amount of metadata to half the
    block size worth of metadata.

    For the sake of description, let us call the number of metadata fields that
    fit inside a block $k$. That is to say, one block can hold the metadata for
    $k$ blocks. The metadata layer discriminates between two types of blocks,
    data blocks and arena header blocks. The metadata layer implements a
    function that takes a block and returns its associated arena header block.
    For our purposes, this function acts as the identity function when passed
    an arena header block. These definitions lead to the notion that an arena
    header block is responsible for storing the metadata of itself, as well as
    the following $k-1$ blocks. The metadata layer is responsible for serving
    the user-facing API.

    See Algorithm \ref{code:mem_meta} for a pseudocode implementation of this
    layer. Note that this layer can also access the variables defined by the
    block layer (they can be found in Algorithm \ref{code:mem_block}). Also note that we have elided the code of several trivial helper functions. For
    example, we did not specify the body of the `get\_header' function. Since
    the implementation of this function was some simple modular arithmetic, we
    did not feel it was worth wasting the reader's time with such minutiae.

    While the metadata layer is parameterized on $k$, for our actual
    implementation, we chose $k$ to be equal to our block size; this implies
    the metadata for each process fits inside 1 byte. Indeed, we stored the
    owner pid of each block as the only metadata, and the bitwidth of the pid
    type in our system was 8.

    \begin{algorithm}
        \caption{Metadata layer pseudocode}
        \label{code:mem_meta}
        \begin{algorithmic}[1]

            \State $k \gets 2^7$
            \Comment{Blocks per arena (configurable)}

            \State $arenaSize \gets k * blockSize$
            \Comment{Arena size (in bytes)}

            \Function{request\_memory\_block}{$pid$}
                \State $header \gets -1$
                \State $blk \gets -1$ \\

                \State $blk \gets \Call{alloc\_block}{ }$
                \If {$blk = -1$}
                    \State \Return $-1$
                \EndIf\\

                \State $header \gets \Call{get\_header}{blk}$
                \If {$blk = header$}
                    \State $blk \gets \Call{alloc\_block}{ }$
                \EndIf\\

                \If {$blk = -1$}
                    \State \Return $-1$
                \EndIf\\

                \State \Call{set\_owner}{$header, blk, pid$}
                \State \Return $blk$
            \EndFunction\\
            \Function{free\_memory\_block}{$blk, pid$}
                \State $header \gets \Call{get\_header}{blk}$
                \If {$header = -1$}
                    \State \Return $-1$
                \EndIf\\

                \If {$ \neg \Call{is\_owner}{header, blk, pid}$}
                    \State \Return $-1$
                \EndIf\\

                \State \Call{set\_owner}{$header, blk, 0$}
                \State \Call{free\_block}{$blk$}
            \EndFunction
        \end{algorithmic}
    \end{algorithm}


\section{Theoretical Analysis}
    Both layers were carefully designed to be $O(1)$ for all operations. This
    should be trivial by inspection, as all the functions do a tiny amount of
    arithmetic, and considering we used a linked-list as the implementation of
    the free list.

    A slightly more interesting analysis is the space overhead of our metadata
    scheme. The metadata layer uses $\frac{k-1}{k}$ blocks for storing user
    data; this yields a storage overhead of $\frac{1}{k}$. For our
    implementation, we picked $k=2^7$, yielding an overhead of $0.78\%$, or a
    utilization rate of $99.22\%$.

\section{Measurements}
    % TODO(sanjay): do some measurements

\chapter{Message Passing}
% Make sure to include pseudocode and testing, if appropriate.

\section{Description}

\section{Running Time Analysis}

\section{Measurements}

\chapter{I/O}
% Make sure to include pseudocode and testing, if appropriate.

\section{UART output}

\section{UART input}

\chapter{Misc}
% Make sure to include pseudocode and testing, if appropriate.

\section{Bridge Layer}

\part{User-level Processes}

\chapter{Proc 1}

\chapter{Proc 2}

\part{Lessons Learned}
% They call this the lessons learned summary.
% 1-2 pages
% what you did do well, both technically and organizationally, and what you
% would do differently if you were to do it again

\end{document}
